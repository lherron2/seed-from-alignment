\documentclass[11pt]{article}
\usepackage[margin=1in]{geometry}
\usepackage{booktabs}
\usepackage{float}
\usepackage{microtype}
\usepackage{pgfplots}
\pgfplotsset{compat=1.18}

\title{RNAnneal-ss top-K output size test (under300 representative50) --- v1}
\date{}

\begin{document}
\maketitle

\section*{Goal}
Measure whether emitting more structures improves oracle best-of-$K$ accuracy, and where diminishing returns start, by evaluating prefixes of a single ranked output list.

\section*{Setup}
\begin{itemize}
  \item \textbf{Targets:} representative50 subset (truth length $\le 300$) from FR3D/BGSU under400.
  \item \textbf{Predictor:} RNAnneal-ss (v3-high settings), run once with \texttt{--top-k 500}.
  \item \textbf{Scaffolds:} EternaFold (Fold + AllSub-like). Duplex sampling uses RNAstructure \texttt{DuplexFold}.
  \item \textbf{Metric:} for each $K \in \{1,50,100,200,500\}$ compute the \emph{oracle} best F1 among the first $K$ outputs.
\end{itemize}

\section*{Summary}
\begin{table}[ht]
\centering
\small
\caption{Overall ablation results (N=27; @1 and best-of-100). Runtime is wall-clock seconds per target.}
\label{tab:overall}
\begin{tabular}{lrrrrr}
\toprule
Config & Mean F1@1 & Mean F1@100 & Fail@100 & Mean time (s) & Med time (s) \\
\midrule
v3 & 0.520 & 0.698 & 0.0\% & 30.8 & 18.9 \\
B4 & 0.505 & 0.698 & 0.0\% & 32.1 & 18.7 \\
U4 & 0.505 & 0.698 & 0.0\% & 32.1 & 19.0 \\
R10 & 0.503 & 0.699 & 0.0\% & 32.1 & 18.1 \\
S150 & 0.505 & 0.698 & 0.0\% & 32.1 & 19.0 \\
K300 & 0.505 & 0.698 & 0.0\% & 32.0 & 18.5 \\
Sc30 & 0.507 & 0.698 & 0.0\% & 33.9 & 18.4 \\
Hi & 0.504 & 0.699 & 0.0\% & 34.0 & 18.1 \\
\bottomrule
\end{tabular}
\end{table}


\section*{Do longer RNAs benefit more?}
\begin{table}[ht]
\centering
\small
\caption{Mean/median $\Delta$F1 (best-of-$K$ minus best-of-1), stratified by RNA length.}
\label{tab:delta_len}
\begin{tabular}{llrrr}
\toprule
Group & K & N & Mean $\Delta$F1 & Med $\Delta$F1 \\
\midrule
short(<=150) & @50 & 23 & 0.109 & 0.068 \\
short(<=150) & @100 & 23 & 0.146 & 0.074 \\
short(<=150) & @200 & 23 & 0.176 & 0.091 \\
short(<=150) & @500 & 23 & 0.233 & 0.099 \\
long(151-300) & @50 & 27 & 0.080 & 0.066 \\
long(151-300) & @100 & 27 & 0.085 & 0.074 \\
long(151-300) & @200 & 27 & 0.105 & 0.086 \\
long(151-300) & @500 & 27 & 0.167 & 0.164 \\
\bottomrule
\end{tabular}
\end{table}

\begin{itemize}
\item Mean $\Delta$F1 @50: short=0.109, long=0.080 (long-short=-0.029).
\item Mean $\Delta$F1 @100: short=0.146, long=0.085 (long-short=-0.061).
\item Mean $\Delta$F1 @200: short=0.176, long=0.105 (long-short=-0.071).
\item Mean $\Delta$F1 @500: short=0.233, long=0.167 (long-short=-0.067).
\item Spearman $\rho$ between truth length and $\Delta$F1 @500: -0.119.
\end{itemize}

\begin{figure}[H]
  \centering
  \begin{tikzpicture}
\definecolor{c_short}{HTML}{1F77B4}
\definecolor{c_long}{HTML}{D62728}
\begin{axis}[
  width=0.9\linewidth, height=0.55\linewidth,
  xmin=0, xlabel={K (structures kept)}, ylabel={Mean $\Delta$F1 vs @1},
  title={Do longer RNAs gain more from larger K?},
  grid=both, grid style={black!10},
  legend pos=north west, legend cell align=left,
  xtick={1,50,100,200,500},
]
\addplot+[very thick, mark=none, draw=c_short] table[x=k,y=delta] {generated/delta_vs_k_short.dat};
\addlegendentry{short (<=150)}
\addplot+[very thick, mark=none, draw=c_long] table[x=k,y=delta] {generated/delta_vs_k_long.dat};
\addlegendentry{long (151-300)}
\end{axis}
\end{tikzpicture}

  \caption{Mean improvement (best-of-$K$ minus best-of-1) for short vs long targets.}
\end{figure}

\clearpage
\section*{Diminishing returns}
\begin{figure}[H]
  \centering
  \begin{tikzpicture}
\begin{axis}[
  width=0.9\linewidth, height=0.55\linewidth,
  xmin=0, xlabel={K (structures kept)}, ylabel={Mean F1},
  title={Mean best-of-K F1 vs output size},
  grid=both, grid style={black!10},
  xtick={1,50,100,200,500},
]
\addplot+[very thick, mark=none] table[x=k,y=f1] {generated/mean_f1_vs_k.dat};
\end{axis}
\end{tikzpicture}

  \vspace{0.5em}
  \begin{tikzpicture}
\begin{axis}[
  width=0.9\linewidth, height=0.55\linewidth,
  xmin=0, ymin=0, ymax=1,
  xlabel={K (structures kept)}, ylabel={Fail rate (F1=0)},
  title={Fail rate vs output size},
  grid=both, grid style={black!10},
  xtick={1,50,100,200,500},
]
\addplot+[very thick, mark=none] table[x=k,y=fail] {generated/fail_vs_k.dat};
\end{axis}
\end{tikzpicture}

  \caption{Mean best-of-$K$ F1 and fail rate as a function of output size.}
\end{figure}

\section*{CDF}
\begin{figure}[H]
  \centering
  \begin{tikzpicture}
\definecolor{c_k1}{HTML}{1F77B4}
\definecolor{c_k50}{HTML}{FF7F0E}
\definecolor{c_k100}{HTML}{2CA02C}
\definecolor{c_k200}{HTML}{D62728}
\definecolor{c_k500}{HTML}{9467BD}
\begin{axis}[
  width=\linewidth, height=0.62\linewidth,
  xmin=0, xmax=1, ymin=0, ymax=1,
  xlabel={F1}, ylabel={CDF},
  title={CDF of best-of-K F1},
  grid=both, grid style={black!10},
  legend pos=north west, legend cell align=left,
  legend style={font=\scriptsize},
]
\addplot+[const plot, very thick, draw=c_k1] table[x=f1,y=cdf] {generated/cdf_1.dat};
\addlegendentry{@1 (p50=0.529)}
\addplot+[const plot, very thick, draw=c_k50] table[x=f1,y=cdf] {generated/cdf_50.dat};
\addlegendentry{@50 (p50=0.615)}
\addplot+[const plot, very thick, draw=c_k100] table[x=f1,y=cdf] {generated/cdf_100.dat};
\addlegendentry{@100 (p50=0.686)}
\addplot+[const plot, very thick, draw=c_k200] table[x=f1,y=cdf] {generated/cdf_200.dat};
\addlegendentry{@200 (p50=0.697)}
\addplot+[const plot, very thick, draw=c_k500] table[x=f1,y=cdf] {generated/cdf_500.dat};
\addlegendentry{@500 (p50=0.743)}
\end{axis}
\end{tikzpicture}

  \caption{Empirical CDF for best-of-$K$ F1. Legend includes the median (p50) F1.}
\end{figure}

\end{document}
